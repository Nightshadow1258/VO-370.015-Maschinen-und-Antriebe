\section{Formelübersicht} % Major section
\subsection{Gleichstrommaschine} % Sub-section
\begin{tabular}{L{2cm}lR{2cm}}
    \textbf{Symbol} &\textbf{Bezeichnung} &\textbf{Einheit}\\
	$I_A$  		&Ankerstrom  			&\si{A}\\
	$U_A$  		&Ankerspannung 			&\si{V}\\
	$R_A$  		&Ankerwiderstand 		&\si{\Omega} \\
	$L_A$		&Ankerinduktivität 		&\si{H}\\
	$\Psi_A$ 	&Ankerfluss 			&\si{Vs}\\
	$I_F$ 		&Feldstrom 				&\si{A}\\
	$U_F$ 		&Feldspannung 			&\si{V}\\
	$R_F$ 		&Feldwiderstand 		&\si{R}\\
	$L_F$ 		& Feldinduktivität 		&\si{H}\\
	$\Psi_F$ 	&Feldfluss 				&\si{Vs}\\
	$U_i$ 		&Induzierte Spannung 		&\si{V}\\
	$\Psi_M$ 	&Permanentmagnetfluss 		&\si{Vs}\\
	$k^{'} \phi$ 	&Spannungskonstante 	&\si{Vs}\\
	$\Omega_m$ 		&Winkelgeschwindigkeit Motor 			&\si{1/s}\\
	$\Omega_{m,0}$ 	&Leerlauf Winkelgeschwindigkeit Motor 	&\si{1/s}\\
	$\Omega_{m,N}$ 	&Nennwinkelgeschwindigkeit Motor 		&\si{1/s}\\
	$\Theta_m$ 		&Trägheitsmoment Motor 					&\si{kg m^{2}}\\
	$M_m$ 		&Moment Motor 			&\si{Nm}\\
	$M_L$ 		&Moment Last 			&\si{Nm}\\
	$A_M$ 		&Fläche Magnet 			&\si{m^{2}}\\
	$A_L$ 		&Fläche Luftspalt 		&\si{m^{2}}\\
	$l_M$ 		&Länge Magnet 			&\si{m}\\
	$l_L$ 		&Länge Luftspalt 		&\si{m}\\
	$B_M$ 		&Flussdichte Magnet 	&\si{T}\\
	$B_r$ 		&Remanenzflussdichte Magnet 	&\si{T}\\
	$B_L$ 		&Flussdichte Luftspalt 			&\si{T}\\
	$\mu_0$ 	&magnetische Feldkonstante 		& \si{Vs / Am}\\
\end{tabular}\\
\subsubsection{Permanentmagneterregt}
Ankerfluss
\begin{equation}
	\Psi_A = \Psi_M + L_A \cdot I_A
\end{equation}
Induzierte Spannung
\begin{equation}
	U_i = k^{'} \phi \cdot \Omega_m \label{glg:induziertespannung}
\end{equation}
Permanentmagnet
\begin{align}
	\Psi_L & = \Psi_M\\
	B_L \cdot A_L & = B_M \cdot A_M\\
	B_L = \mu_0 \cdot H_L & = B_M \frac{A_M}{A_L}\\
	2 \cdot H_L \cdot l_L & = 2 \cdot H_M \cdot l_M\\
	H_M & = H_L \cdot \frac{l_L}{l_M}\\
	B_M & = B_r + \mu_0 \cdot \mu_r \cdot H_M\\
	B_M & = B_r + \mu_0 \cdot \mu_r \cdot H_L \cdot \frac{l_L}{l_M}\\
	B_M & = B_r +  \mu_r \cdot B_M \frac{A_M}{A_L} \cdot \frac{l_L}{l_M}\\
	B_M & = \frac{B_r}{1- \mu_r  \frac{A_M}{A_L} \cdot \frac{l_L}{l_M}}
\end{align}
Ankerspannungsgleichung
\begin{align}
	U_A &=R_A  I_A + \frac{\partial \Psi_A }{\partial t}  \\
	&= R_A  I_A + \frac{\partial \Psi_M + L_A  I_A }{\partial t}\\
	&= R_A  I_A + L_A  \frac{\partial I_A }{\partial t} + k^{'} \phi \cdot \Omega_m\label{glg:Ankerspannungsgleichung}
\end{align}
Moment des Motors
\begin{equation}
	M_m = k^{'} \phi \cdot I_A \label{glg:Ankermoment}
\end{equation}
Mechanische Gleichung
\begin{align}
	\Theta_m \frac{\partial \Omega_m}{\partial t} &= M_m - M_L \\
	\Theta_m \frac{\partial \Omega_m}{\partial t} &= k^{'} \phi \cdot I_A - M_L\label{glg:mechanische}
\end{align}
Mechanische Leistung
\begin{equation}
	P_{mech} =M_m \cdot \Omega_m = k^{'} \phi \cdot I_A \cdot \Omega_m\label{glg:mechanischleistung}
\end{equation}
Wirkungsgrad
\begin{equation}
	\eta_N = \frac{\Omega_N M_N}{U_N I_N}\label{glg:Wirkungsgrad}
\end{equation}
Laplace Bereich
\begin{align}
	U_A(s)&= R_A  I_A(s) + L_A   I_A(s)s + k^{'} \phi \cdot \Omega_m(s) \\
	\Theta_m  \Omega_m(s) s &= M_m - M_L=k^{'} \phi \cdot I_A(s) - k_L \cdot \Omega_m(s) \\
	\frac{\Omega_m(s)}{U_A(s)} &= \frac{k^{'} \phi}{s^{2} L_A \Theta_m + s R_A \Theta_m + (k^{'} \phi)^{2}}\\
	\frac{\Omega_m(s)}{M_L(s)} &= \frac{R_A + s L_A}{s^{2} L_A \Theta_m + s R_A \Theta_m + (k^{'} \phi)^{2}} \\
	\frac{\Omega_m(s)}{I_A(s)} &= \frac{k^{'} \phi}{ s  \Theta_m + k_L}
\end{align}
Im Stationären Fall gilt folgendes:
\begin{equation}
	\frac{\partial \Omega_m}{\partial t} = \frac{\partial I_A }{\partial t} = 0
\end{equation}
\subsubsection{Fremderregt}
\begin{align}
	U_F &= I_F R_F + \frac{\partial \Psi_F}{\partial t}\\
	\Psi_F &= L_F(I_F) I_F
\end{align}
\subsection{Permanentmagneterregte Synchronmaschine}
\begin{tabular}{L{2cm}lR{2cm}}
	$\underline{i}_s$  &bezogener Statorstrom statorfest &[$1$]\\
	$\underline{i}_{sdq}$  &bezogener Statorstrom rotorfest &[$1$]\\
	$\underline{u}_s$ &bezogene Statorspannung statorfest & [$1$]\\
	$\underline{u}_{sdq}$ &bezogene Statorspannung rotorfest & [$1$]\\
	$r_s$  &bezogener Statorwiderstand &[$1$] \\
	$\omega_K$  &bezogenes Rotierendes Koordinatensystem &[$1$]\\
	$\omega_m$ &bezogene Winkelgeschwindigkeit Motor & [$1$]\\
	$l_s$ &bezogene Statorinduktivität & [$1$]\\
	$U_i$ &bezogene Induzierte Spannung & [$1$]\\
	$\underline{\Psi}_M$ &bezogener Permanentmagnetfluss & [$1$]\\
	$\underline{\Psi}_s$ &bezogene Statorflussverkettung & [$1$]\\
	$\tau$ &bezogene Zeit & [$1$]\\
	$\tau_m$ &bezogenes Trägheitsmoment Motor & [$1$]\\
	$m_R$ &bezogenes Moment Rotor & [$1$]\\
	$m_L$ &bezogenes Moment Last & [$1$]\\
\end{tabular}\\
\subsubsection{Wechselstrombetrieb}
Statorinduktivität
\begin{equation}
	l_s = \frac{3}{2} \cdot l_{strang}
\end{equation}
Statorflussverkettungsgleichung
\begin{equation}
	\underline{\Psi}_s = l_s \cdot \underline{i}_s + \underline{\Psi}_M = l_s \cdot \underline{i}_s + |\underline{\Psi}_M | \cdot e^{\jmath \cdot \gamma + \jmath \omega \tau}
\end{equation}
Rotorstrom
\begin{align}
	\arg(\underline{i}_{sdq}) &= \arg(\underline{i}_s) -\arg(\underline{\Psi}_M)\\
	\underline{i}_{sdq} &= |\underline{i}_s| \cdot e^{\jmath \arg(\underline{i}_{sdq})}\\
	\underline{i}_{sd} & = |\underline{i}_{sdq}| \cdot \cos(\arg(\underline{i}_{sdq}))\\
	\underline{i}_{sq} & = |\underline{i}_{sdq}| \cdot \sin(\arg(\underline{i}_{sdq}))\\
\end{align}
Statorstrom
\begin{equation}
	\underline{i}_{s} = |\underline{i}_{sdq}| \cdot e^{\jmath (\arg(\underline{i}_{sdq}) + \arg(\underline{\Psi}_{M}))}
\end{equation}
Statorspannungsgleichung
\begin{equation}
	\underline{u}_s(\tau) = \underline{i}_s \cdot r_s + \frac{\partial \underline{\Psi}_s}{\partial \tau} + \jmath \omega_K \cdot \underline{\Psi}_s
\end{equation}
Statorspannungsgleichung im rotorfesten Koordinatensystem
\begin{align}
	\underline{u}_{sdq}(\tau) &= \underline{i}_{sdq} \cdot r_s + l_s \cdot \frac{\partial \underline{i}_{sdq}}{\partial \tau} + \frac{\partial |\underline{\Psi}_M|}{\partial \tau} + \jmath \omega_K \cdot l_s \cdot \underline{i}_{sdq} + \jmath \omega_K \cdot |\underline{\Psi}_M|\label{glg:Statorspannungraumzeiger rotor} \\
	\frac{\partial |\underline{\Psi}_M|}{\partial \tau} & = 0 \label{glg:rotorfestmagnetisch} \\
	\omega_K=\omega_m \label{glg:rotorfestgeschwindigkeit} \\
	\underline{u}_{sdq}(\tau) &= \underline{i}_{sdq} \cdot r_s + l_s \cdot \frac{\partial \underline{i}_{sdq}}{\partial \tau} + \jmath \omega_m \cdot l_S \cdot \underline{i}_{sdq} + \jmath \omega_m \cdot |\underline{\Psi}_M|
\end{align}
Stromzeiger im rotorfesten Koordinatensystem bei Kurzschluss im stationären Fall
\begin{align}
	0 &= \underline{i}_{sdq} \cdot r_s + \jmath \omega_m \cdot l_S \cdot \underline{i}_{sdq} + \jmath \omega_m \cdot |\underline{\Psi}_M|\\
	\underline{i}_{sdq} \cdot (r_s + \jmath \omega_m \cdot l_s) & =  - \jmath \omega_m \cdot |\underline{\Psi}_M|\\
	\underline{i}_{sdq}  & =  \frac{- \jmath \omega_m \cdot |\underline{\Psi}_M|}{(r_s + \jmath \omega_m \cdot l_s)}\\
	\underline{i}_{sdq}  & =  \frac{- \jmath \omega_m \cdot |\underline{\Psi}_{M}|}{(r_s + \jmath \omega_m \cdot l_s)} \cdot \frac{(r_s - \jmath \omega_m \cdot l_s)}{(r_s - \jmath \omega_m \cdot l_s)}\\
	\underline{i}_{sdq}  & = \frac{-\omega^2 |\underline{\Psi}_M| l_s}{r_s^2 + (\omega l_s)^2} -\jmath \frac{\omega |\underline{\Psi}_M| r_s}{r_s^2 + (\omega l_s)^2}\label{glg:stromzeiger rotorfest ks}
\end{align}
Im rotorfesten Koordinatensystem ist die zeitliche Änderung des magnetischen Flusses vom Permanentmagneten null Glg.(\ref{glg:rotorfestmagnetisch}), weil  sich der Magnet mit dem Koordinatensystem bewegt. Das Koordinatensystem bewegt sich mit der Geschwindigkeit des Motors. Glg.(\ref{glg:rotorfestgeschwindigkeit})\\ 
Statorspannungsgleichung im statorfesten Koordinatensystem
\begin{align}
	\arg(\underline{i}_s) &= \arg(\underline{i}_{sdq}) + \arg(\underline{\Psi}_M)\\
	\underline{i}_s &= |\underline{i}_{sdq}| \cdot e^{\jmath \arg(\underline{i}_{s})}\\
	\underline{u}_s(\tau) &= \underline{i}_s \cdot r_s + l_s \cdot \frac{\partial \underline{i}_s}{\partial \tau} + \frac{\partial \underline{\Psi}_M}{\partial \tau} + \jmath \omega_K \cdot l_s \cdot \underline{i}_s + \jmath \omega_K \cdot \underline{\Psi}_M \\
	\omega_K=0 \label{glg:statorfest} \\
	\frac{\partial \underline{\Psi}_M}{\partial \tau} &= \frac{\partial |\underline{\Psi}_M | \cdot e^{\jmath \cdot \gamma + \jmath \omega \tau}}{\partial \tau} = \jmath\omega \cdot|\underline{\Psi}_M | \cdot e^{\jmath \cdot \gamma + \jmath \omega \tau}\\
	\underline{u}_s(\tau) &= \underline{i}_s \cdot r_s + l_s \cdot \frac{\partial \underline{i}_s}{\partial \tau} + \jmath\omega \cdot|\underline{\Psi}_M | \cdot e^{\jmath \cdot \gamma + \jmath \omega \tau} \label{glg:spannungraumzeiger rotor}
\end{align}
Statorstromzeigers im statorfesten Koordinatensystem bei Kurzschluss im stationären Fall
\begin{align}
	0 &= \underline{i}_s \cdot r_s + \jmath\omega \cdot|\underline{\Psi}_M | \cdot e^{\jmath \cdot \gamma + \jmath \omega \tau} \\
	\underline{i}_s &= \frac{\jmath\omega \cdot|\underline{\Psi}_M | \cdot e^{\jmath \cdot \gamma + \jmath \omega \tau}}{r_s}
\end{align}
\begin{comment}
	Statorkoordinatensystem [KOS]
	\begin{figure}
		\begin{tikzpicture}
			\begin{polaraxis}[ymax = 1.3]
				\addplot [->,blue,no marks,very thick ]coordinates { (0,0) (-15,1)};
				\addplot [->,blue,no marks,very thick ]coordinates { (0,0) (75,1)};
				\addplot [->,red,no marks,very thick ]coordinates { (0,0) (0,1)};
				\addplot [->,red,no marks,very thick ]coordinates { (0,0) (90,1)};
				\node[] (x) at (axis cs:-5,1.15){$\alpha$(Stator)};
				\node[] (q) at (axis cs:-20,1.15){q (Rotor)};
				\node[] (y) at (axis cs:95,1.15){$\beta$};
				\node[] (d) at (axis cs:70,1.15){d};
			\end{polaraxis}
		\end{tikzpicture}
	\end{figure}
\end{comment}
Drehmomentgleichung
\begin{equation}
	m_R(\tau) = -Im(\underline{i}_s^{*} \cdot \underline{\Psi}_s) = i_{sq} \cdot | \underline{\Psi}_M| \label{glg:synmoment}
\end{equation}
Mechanische Gleichung
\begin{equation}
	\tau_m \cdot \frac{\partial \omega_m}{\partial \tau} = m_R - m_L= i_{sq} \cdot | \underline{\Psi}_M| - m_L
\end{equation}
Strangströme
\begin{align}
	{i}_1 & = \Re \{ \underline{i}_s \cdot e^{\jmath \cdot 0 ^\circ} \} =|\underline{i}_s| \cdot \cos(\arg(\underline{i}_s)) \label{glg:strang1}\\
	{i}_2 & = \Re \{ \underline{i}_s \cdot e^{-\jmath \cdot 120 ^\circ} \} =|\underline{i}_s| \cdot \cos(\arg(\underline{i}_s)-120) \label{glg:strang2}\\
	{i}_3 & = \Re \{ \underline{i}_s \cdot e^{\jmath \cdot 120 ^\circ} \} =|\underline{i}_s| \cdot \cos(\arg(\underline{i}_s)+120) \label{glg:strang3}
\end{align}
\subsubsection{BLDC-Betrieb}
Stromraumzeiger\\
\begin{table}
	\caption{Stromzeiger im BLDC Betrieb} \label{tab:bldc}
	\begin{tabular}{|c|c|c|c|c|}\hline 
		Name & $i_1$ & $i_2$ & $i_3$ & Winkel [$^\circ$] \\ \hline
		A & + & - & 0 & -30 \\ \hline
		B & + & 0 & - & 30 \\ \hline
		C & 0 & + & - & 90 \\ \hline
		D & - & + & 0 & 150 \\ \hline
		E & - & 0 & + & 210 \\ \hline
		F & 0 & - & + & 270 \\ \hline
	\end{tabular}\\
\end{table}
\begin{figure}[H]
	\begin{tikzpicture}
		\begin{polaraxis}[ymax = 1.3]
			\addplot [->,blue,no marks,very thick ]coordinates { (0,0) (-30,1)}node[pos=1,anchor=north west]{$A$};
			\addplot [->,blue,no marks,very thick ]coordinates { (0,0) (30,1)}node[pos=1,anchor=south west]{$B$};
			\addplot [->,blue,no marks,very thick ]coordinates { (0,0) (90,1)}node[pos=1,anchor=south]{$C$};
			\addplot [->,blue,no marks,very thick ]coordinates { (0,0) (150,1)}node[pos=1,anchor=south east]{$D$};
			\addplot [->,blue,no marks,very thick ]coordinates { (0,0) (210,1)}node[pos=1,anchor=north east]{$E$};
			\addplot [->,blue,no marks,very thick ]coordinates { (0,0) (270,1)}node[pos=1,anchor=north]{$F$};
		\end{polaraxis}
	\end{tikzpicture}
	\caption{Stromzeiger graphisch dargestellt} \label{fig:bldc}
\end{figure}
\begin{equation}
	\underline{i}_s = \frac{2}{3} \cdot [ i_{1} + i_{2} \cdot e^{\jmath \cdot 120 ^\circ} + i_{3} \cdot e^{\jmath \cdot 240 ^\circ}]\label{glg:BLDC-Stromzeiger}
\end{equation} 
Für den Fall A, also wenn der Stromzeiger bei $-30 ^\circ$ steht, wird in die Glg.(\ref{glg:BLDC-Stromzeiger}) für $i_{1}= 1$, $i_{2}= -1$ und $i_{3}= 0$ eingesetzt.