\begin{question}[section=1,name={21.1.2015},mode=exm,type=bsp,tags={20150121}]
Eine dreisträngige symetrisch aufgebaute permanentmagneterregte Synchronmaschine in Y-Schaltung mit $I_N = 12~A$ (Effektivwert) habe zu dem betrachteten Zeitpunkt $\tau_0$ einen normierten Rotorverkettungsfluss von $\underline{\Psi}_M = 1 \cdot e^{\jmath 40 ^\circ}$. Zu diesem Zeitpunkt $\tau_0$ ist der normierte statorfeste Stromraumzeiger $\underline{i}_s = 0,5 \cdot e^{\jmath 110 ^\circ}$.
\begin{enumerate}
\item Berechnen Sie für diesen Zeitpunkt $\tau_0$ die bezogenen Strangströme $i_1$, $i_2$ und $i_3$ sowie die nicht bezogenen Ströme $I_1$, $I_2$ und $I_3$. (\addpoints{2})\label{2015012111}
\item Berechnen Sie für den Zeitpunkt $\tau_0$ den normierten Stromraumzeiger im rotorfesten Koordinatensystem und das bezogene Drehmoment der Maschine. Skizzieren Sie die Raumzeiger $\underline{\Psi}_M$ und $\underline{i}_s$ sowie die dem Moment entsprechende Fläche in der komplexen Raumzeigerebene (Strangachse ``U'' liegt in der reellen Achse) des obrigen Betriebpunktes. (\addpoints{2})
\item Berechnen Sie alternativ für den BLDC-Betrieb zum Zeitpunkt $\tau_0$ jenen normierten Stromraumzeiger im statorfesten Koordinatensystem, welcher das \textbf{halbe motorische} Bezugsmoment bei positiver Drehrichtung ergibt. (Die Berechnung soll unter optimaler Drehmomentenausbeute erfolgen). Geben Sie ebenfalls die bezogenen Ströme $i_1$, $i_2$ und $i_3$ in den Motorzuleitungen an. (\addpoints{3})
\item Skizzieren Sie die Raumzeiger $\underline{\Psi}_M$ und $\underline{i}_s$ sowie die dem Moment entsprechende Fläche in der komplexen Raumzeigerebene (Strangachse ``U'' liegt in der reellen Achse) des obrigen Betriebpunktes. (\addpoints{1})
\item Berechnen Sie den bezogenen rotorfesten Spannungsraumzeiger für den Sinus-Betrieb (ensprechend dem Punkt \ref{2015012111}.) zum Zeitpunkt $\tau_0$, wenn die Maschine gerade mit 20\% der der Bezugsdrehzahl rotiert. Verwenden Sie dazu die Maschinenparameter $r_s = 0,07$ und $l_s= 0,25$. (\addpoints{2})
\end{enumerate}
\end{question}
\begin{solution}
\begin{enumerate}
\item  Da zum Zeitpunkt $\tau_0$ der Rotorverkettungsfluss dem Stromraumzeiger nacheilt, kann es sich in diesem Beispiel nur um ein linksdrehenden Motor oder um einen rechtsdrehenden Generator handeln. 
In die Glg.(\ref{glg:strang1}),(\ref{glg:strang2}) und (\ref{glg:strang3}) wird der Statorstrom $\underline{i}_s$ eingesetzt.
\begin{align}
i_1 & = \Re \{ \underline{i}_s \cdot e^{\jmath \cdot 0 ^\circ} \} = -0,171\\
i_2 & = \Re \{ \underline{i}_s \cdot e^{-\jmath \cdot 120 ^\circ} \} = 0,492 \\
i_3 & = \Re \{ \underline{i}_s \cdot e^{\jmath \cdot 120 ^\circ} \}=  -0,321
\end{align}
Um die nicht bezogenen Ströme zu erhalten werden die bezogenen Ströme mit dem Bezugswert $I_N \cdot \sqrt{2}$ multipliziert. (Effektivwert auf Spitzenwert umrechnen)
\begin{align}
I_1 & = i_1 \cdot I_N \cdot \sqrt{2} = -0,171 \cdot 12 A \cdot \sqrt{2} =-2,902~A \\
I_2 & = i_2 \cdot I_N \cdot \sqrt{2} = 0,492 \cdot 12 A \cdot \sqrt{2} =8,356~A \\
I_3 & = i_3 \cdot I_N \cdot \sqrt{2} =-0,321 \cdot 12 A \cdot \sqrt{2} =-5,454~A
\end{align}
\item Im Rotorfesten Koordinatensystem ist der Stromzeiger um $40^\circ$ in negativer Drehrichtung verschoben. Der Statorstrom wird auch gleich in seine Komponente $\underline{i}_{sq}$ und $\underline{i}_{sd}$ aufgeteilt. Dann wird $\underline{i}_{sq}$ und $\Psi_M$ in Glg.(\ref{glg:synmoment}) eingesetzt.
\begin{align}
\underline{i}_{sdq} & = \underline{i}_s \cdot e^{-\jmath 40 ^\circ} = 0,5 \cdot e^{\jmath 110 ^\circ} \cdot e^{-\jmath 40 ^\circ} = 0,5 \cdot e^{\jmath 70 ^\circ} \\
\underline{i}_{sd} & = |\underline{i}_{sdq}| \cdot \cos(\arg(\underline{i}_{sdq})) = 0,5 \cdot \cos(70) = 0,171 \\
\underline{i}_{sq} & = |\underline{i}_{sdq}| \cdot \sin(\arg(\underline{i}_{sdq})) = 0,5 \cdot \sin(70) = 0,47 \\
m_R(\tau)& =  i_{sq} \cdot | \underline{\Psi}_M|= 0,47\cdot 1 = 0,47
\end{align}
Da das Moment positiv ist, ist hier ersichtlich, dass es sich um einen motorischen linksbetrieb handelt.
\textbf{TODO:} Programmiere oder zeichne die Grafik und scan sie ein und lade sie hoch!
\item Das \textbf{halbe motorische} Bezugsmoment bei positiver Drehzahl bedeutet $m_m = 0,5$. Im BLDC Betrieb ist der nächste Stromzeiger gem Abb.(\ref{fig:bldc}) zu $40^\circ$ bei $150^\circ$ der Fall D. Über die Dreiecksbeziehungen werden $\underline{i}_{sdq}$ und $\underline{i}_{sd}$ aus $\underline{i}_{sq}$ berechnet. Anschließend wird $\underline{i}_{sdq}$ auf $\underline{i}_{s}$ umgeformt um die Ströme in den Motorzuleitungen zu berechnen.
\begin{align}
m_m &= 0,5 = \underline{i}_{sq} \cdot |\underline{\Psi_m}|\\
0,5 &= \underline{i}_{sq} \cdot 1\\
\underline{i}_{sq} &= 0,5\\
\arg(\underline{i}_{sdq}) &= \arg(\underline{i}_{s}) -\arg(\underline{\Psi}_{M})=110^\circ\\
\underline{i}_{sdq} &= \frac{\underline{i}_{sq}}{\sin(\arg(\underline{i}_{sdq}))}= 0,532\\
\underline{i}_{sd} &= \underline{i}_{sdq} \cdot \cos(\arg(\underline{i}_{sdq})) = -0,182\\
\underline{i}_{s} &= |\underline{i}_{sdq}| \cdot e^{\jmath (\arg(\underline{i}_{sdq}) + \arg(\underline{\Psi}_{M}))}= 0,532 \cdot e^{\jmath ( 110 + (40))}
\end{align}
In die Glg.(\ref{glg:strang1}),(\ref{glg:strang2}) und (\ref{glg:strang3}) wird der Statorstrom $\underline{i}_s$ eingesetzt.
\begin{align}
i_1 & = \Re \{ \underline{i}_s \cdot e^{\jmath \cdot 0 ^\circ} \} = -0,461\\
i_2 & = \Re \{ \underline{i}_s \cdot e^{-\jmath \cdot 120 ^\circ} \} = 0,461 \\
i_3 & = \Re \{ \underline{i}_s \cdot e^{\jmath \cdot 120 ^\circ} \}=  0
\end{align}
Wie in Tab.(\ref{tab:bldc}) ersichtlich muss für den Fall D $i_1= -i_2$ sein und $i_3= 0$ gelten.
\item \textbf{TODO:} Programmiere oder zeichne die Grafik und scan sie ein und lade sie hoch!
\item Der Spannungraumzeiger im rotorfesten Koordinatensystem errechnet sich nach Glg.(\ref{glg:Statorspannungraumzeiger rotor}), wobei die partiellen Terme wegfallen, weil wir uns im stationären Fall befinden. Für $\underline{i}_{sdq}= 0,5 e^{\jmath 70^\circ}$ einsetzen.
\begin{align}
\underline{u}_{sdq}(\tau) &= \underline{i}_{sdq} \cdot r_s + l_s \cdot \frac{\partial \underline{i}_{sdq}}{\partial \tau} + \frac{\partial |\underline{\Psi}_M|}{\partial \tau} + \jmath \omega_K \cdot l_s \cdot \underline{i}_{sdq} + \jmath \omega_K \cdot |\underline{\Psi}_M|\\
\underline{u}_{sdq}(\tau) &= \underline{i}_{sdq} \cdot r_s + \jmath \omega_K \cdot l_s \cdot \underline{i}_{sdq} + \jmath \omega_K \cdot |\underline{\Psi}_M|\\
&= 0,5 e^{\jmath 70^\circ} \cdot 0,07 + \jmath 0,2 \cdot 0,25 \cdot 0,5 e^{\jmath 70^\circ}+\jmath 0,2 \cdot 1\\
&=0,2 \cdot e^{\jmath 90,47^\circ}
\end{align}
\end{enumerate}
\end{solution}