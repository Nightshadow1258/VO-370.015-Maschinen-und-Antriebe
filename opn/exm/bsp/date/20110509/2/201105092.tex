\begin{question}[section=2,name={9.5.2011},mode=exm,type=bsp,tags={20110509}]
Eine Maschine ist über ein Getriebe an die Seiltrommel eines Krans gekuppelt. Das Übersetzungsverhältnis beträgt 100:1, die Seiltrommel hat einen Radius von $75~cm$. Die maximale Last des Krans beträgt $5~t$. Die Fremderregte Gleichstrommaschine hat folgende Daten.\\
\begin{tabular}{L{2cm}l}
$I_{N}$ \dotfill &$200~A$\\
$k \phi_N$ \dotfill & $15~Vs$ \\
$n_N$ \dotfill & $1960~\frac{U}{min}$\\
$n_0$ \dotfill & $2000~\frac{U}{min}$
\end{tabular}
\begin{enumerate}
\item Berechnen Sie den Ankerwiderstand der Gleichstrommaschine. (\addpoints{2})
\item Der Motor wird mit einer konstanten Spannung von $500~V$ versorgt. Wie groß ist ein Vorwiderstand $R_V$ zu wählen, damit die vorerst ruhende Last mit der Mindestbeschleunigung $1~ m/s^2$ gehoben wird? (\addpoints{3})
\item Welche maximale Geschwindigkeit der Last ist mit diesem Vorwiderstand erreichbar? (\addpoints{3})
\item Zeichnen Sie schematisch den zeitlichen Verlauf des Ankerstromes sowie der Geschwindigkeit der Last über die Zeit. (\addpoints{2})
\end{enumerate}
\end{question}
\begin{solution}
\textbf{Hinweis:} Diese Prüfung ist noch aus dem Masterstudium und ist nicht repräsentativ für den Prüfungsstoff im Bachelorstudium.
\end{solution}