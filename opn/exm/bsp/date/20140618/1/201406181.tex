\begin{question}[section=1,name={18.6.2014},mode=exm,type=bsp,tags={20140618}]
Eine dreisträngige symetrische aufgebaute permanentmagneterregte Synchronmaschine habe zu dem betrachteten Zeitpunkt einen normierten Rotorverkettungsfluss von $\underline{\Psi}_M = 1 \cdot e^{-\jmath15^\circ}$. Folgende Berechnungen sollen unter optimaler Drehmomentausbeute erfolgen.
\begin{enumerate}
\item Berechnen Sie für einen BLDC-Betrieb jenen normierten Stromanzeiger im statorfesten Koordinatensystem, welcher das motorische Bezugsmoment bei positiver Drehrichtung ergibt. Geben Sie ebenfalls die bezogenen Ströme $i_1$, $i_2$ und $i_3$ in den Motorzuleitungen an. (\addpoints{3})
\item Skizzieren Sie $\underline{\Psi}_M$, $\underline{i}_s$ und die dem Moment entsprechende Fläche in der komplexen Raumzeigerebene (Strangachse ``U'' liegt in der reellen Achse) des obrigen Betriebpunktes. (\addpoints{1})
\item Berechnen Sie für einen Sinus-Betrieb jenen normierten Stromraumzeiger im statorfesten Koordinatensystem, welcher das morotrische Bezugsmoment bei positiver Drehrichtung ergibt. Geben Sie ebenfalls die bezogenen Ströme $i_1$, $i_2$ und $i_3$ in den Motorzuleitungen an. (\addpoints{3})
\item Skizzieren Sie $\underline{\Psi}_M$, $\underline{i}_s$ und die dem Moment entsprechende Fläche in der komplexen Raumzeigerebene (Strangachse ``U'' liegt in der reellen Achse) des obrigen Betriebspunktes. (\addpoints{1})
\end{enumerate}
\end{question}
\begin{solution}
\begin{enumerate}
\item Das motorische Bezugsmoment bei positiver Drehzahl bedeutet $m_m = 1$. Im BLDC Betrieb ist der nächste Stromzeiger gem Abb.(\ref{fig:bldc}) zu $-15^\circ$ bei $90^\circ$ der Fall C. Über die Dreiecksbeziehungen werden $\underline{i}_{sdq}$ und $\underline{i}_{sd}$ aus $\underline{i}_{sq}$ berechnet. Anschließend wird $\underline{i}_{sdq}$ auf $\underline{i}_{s}$ umgeformt um die Ströme in den Motorzuleitungen zu berechnen.
\begin{align}
m_m &= 1 = \underline{i}_{sq} \cdot |\underline{\Psi_m}|\\
1 &= \underline{i}_{sq} \cdot 1\\
\underline{i}_{sq} &= 1\\
\arg(\underline{i}_{sdq}) &= \arg(\underline{i}_{s}) -\arg(\underline{\Psi}_{M})=105^\circ\\
\underline{i}_{sdq} &= \frac{\underline{i}_{sq}}{\sin(\arg(\underline{i}_{sdq}))}= 1,0353\\
\underline{i}_{sd} &= \underline{i}_{sdq} \cdot \cos(\arg(\underline{i}_{sdq})) = -0,267\\
\underline{i}_{s} &= |\underline{i}_{sdq}| \cdot e^{\jmath (\arg(\underline{i}_{sdq}) + \arg(\underline{\Psi}_{M}))}= 1,0353 \cdot e^{\jmath ( 105 + (-15))}
\end{align}
In die Glg.(\ref{glg:strang1}),(\ref{glg:strang2}) und (\ref{glg:strang3}) wird der Statorstrom $\underline{i}_s$ eingesetzt.
\begin{align}
i_1 & = \Re \{ \underline{i}_s \cdot e^{\jmath \cdot 0 ^\circ} \} = 0\\
i_2 & = \Re \{ \underline{i}_s \cdot e^{-\jmath \cdot 120 ^\circ} \} = 0,897 \\
i_3 & = \Re \{ \underline{i}_s \cdot e^{\jmath \cdot 120 ^\circ} \}=  -0,897
\end{align}
Wie in Tab.(\ref{tab:bldc}) ersichtlich muss für den Fall C $i_2= -i_3$ sein und $i_1= 0$ gelten.
\item \textbf{TODO:} Programmiere oder zeichne die Grafik und scan sie ein und lade sie hoch!
\item Da wir uns im Sinus-Betrieb befinden ist der Stromraumzeiger $\underline{i}_{sq}$, welche das optimale Bezugsdrehmoment liefert, gleich dem Stromraumzeiger $\underline{i}_{sdq}$. Der Winkel liegt somit exakt bei $\arg(\underline{i}_{sdq})=90^\circ$ zu $\underline{\Psi}_M$.
\begin{align}
m_m &= 1 = \underline{i}_{sq} \cdot |\underline{\Psi_m}|\\
1 &= \underline{i}_{sq} \cdot 1\\
\underline{i}_{sq} &= 1 = \underline{i}_{sdq} \\
\underline{i}_{sd} &= 0\\
\underline{i}_{s} &= |\underline{i}_{sdq}| \cdot e^{\jmath (\arg(\underline{i}_{sdq}) + \arg(\underline{\Psi}_{M}))}= 0,5 \cdot e^{\jmath (90 - 15)}
\end{align}
In die Glg.(\ref{glg:strang1}),(\ref{glg:strang2}) und (\ref{glg:strang3}) wird der Statorstrom $\underline{i}_s$ eingesetzt.
\begin{align}
i_1 & = \Re \{ \underline{i}_s \cdot e^{\jmath \cdot 0 ^\circ} \} = 0,259\\
i_2 & = \Re \{ \underline{i}_s \cdot e^{-\jmath \cdot 120 ^\circ} \} = 0,707 \\
i_3 & = \Re \{ \underline{i}_s \cdot e^{\jmath \cdot 120 ^\circ} \}=  -0,966
\end{align}
\item \textbf{TODO:} Programmiere oder zeichne die Grafik und scan sie ein und lade sie hoch!
\end{enumerate}
\end{solution}