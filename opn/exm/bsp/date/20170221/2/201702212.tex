\begin{question}[section=2,name={21.2.2017},mode=exm,type=bsp,tags={20170221}]
	Eine kompensierte \textbf{Reihenschluss-Gleichstrommaschine} hat folgende Daten .
	\begin{tabular}{L{2cm}l}
		$I_{A,N}$ \dotfill &$225~A$\\
		$P_{N,mech}$ \dotfill & $81~kW$ \\
		$n_N$ \dotfill & $2000~\frac{U}{min}$\\
		$\mu_{N}$ \dotfill & $90~\%$\\
	\end{tabular}\\
	Die Maschine ist im Nennpunkt nicht gesättigt ($\Phi \approx I_A$). Die Maschine hat nur ohmsche Verluste, Reibungsverluste und Eisenverluste sind vernachlässigbar.
	\begin{enumerate}
		\item Skizzieren Sie die Schaltung der Reihenschlussmaschine am Gleichspannungsnetz inkl. aller Widerstände und Induktivitäten der Maschine. (\addpoints{1})
		\item Wie groß ist das Nennmoment $M_N$ und die Spannungskonstante $k_1 \phi_N$ im Nennpunkt im motorischen Betrieb? Wie groß ist die Leerlaufdrehzahl $n_0$? (\addpoints{2})
		\item Wie groß ist die Nennspannung $U_N$ und wie groß ist der Ankerwiderstand $R_A$ und der Erregerwiderstand $R_E$, wenn sich der Erregerwiderstand $R_E$ zum Ankerwiderstand $R_E:R_A = 2:5$ verhält? (\addpoints{2})
		\item Skizzieren Sie maßstäblich die Drehzahl-Drehmoment Kennlinie (M/n) bei Nennspannung im Bereich ca. $0,2 M_N$ bis $1,5 M_N$. (\addpoints{2})
		\item Berechnen Sie den benötigten Vorwiderstand $R_V$, wenn die Maschine bei Nennspannung $U_N$ mit $M=1,5 \cdot M_N$ aus dem Stillstand angefahren werden soll. (\addpoints{1})
		\item Die Gleichstrommaschine wird von einem Stromrichter mit konstantem, halben Nennstrom bei $n=2000~U/min$ als Motor betrieben. Berechnen und Skizzieren Sie den Drehzahlverlauf $n(t)$, wenn die Last schlagartig abgekuppelt wird. Der Stromrichter liefert dabei weiterhin den konstanten Strom und schaltet die Gleichstrommaschine erst bei Erreichen einer Spannung von $U=460~V$ ab. Wie lange dauert es bis zum Abschalten und welche Enddrehzahl wird erreicht? Das Trägheitsmoment der Gleichstrommaschine ist $\Theta_{GM} = 12~kgm^2$ und das Reibmoment beträgt konstant $1 \%$ des Nennmoments. (\addpoints{2})
	\end{enumerate}
\end{question}
\begin{solution}
	
\end{solution}