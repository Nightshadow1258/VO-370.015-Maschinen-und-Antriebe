\begin{question}[section=1,name={18.1.2017},mode=exm,type=bsp,tags={20170118,20170221}]
	Eine dreisträngige, vierpolige (2p=4) symetrische aufgebaute permanentmagneterregte Synchronmaschine in Y-Schaltung hat die Nennwerte (Effektivwerte):\\
	\begin{tabular}{L{2cm}l}
		$I_{N}$ \dotfill &$8~A$\\
		$U_{N}$ \dotfill & $400~V$ \\
		$n_N$ \dotfill & $2250~\frac{U}{min}$\\
	\end{tabular}
	\begin{compactenum}
		\item Berechnen Sie den Bezugsstrom $I_{Bez}$, die Bezugsspannung $U_{Bez}$, den Bezugswiderstand $R_{Bez}$, die Bezugskreifrequenz $\Omega_{Bez}$, die Bezugszeit $T_{Bez}$ und die Bezugsinduktivität $L_{Bez}$. (\addpoints{2})
		\item Berechnen Sie für den Zeitpunkt $\tau_0$, den optimalen \underline{statorfesten und rotorfesten} Stromraumzeiger im \textbf{BLDC-Betrieb} für ein Drehmoment $m=2/3$, wenn zum Zeitpunkt $\tau_0$ der normierte statorfeste Rotorverkettungsfluß $\underline{\Psi}_M = 1 \cdot e^{\jmath 40^\circ}$ beträgt. Skizzieren Sie maßstäblich die Raumzeiger $\underline{\Psi}_M$ und $\underline{i}_s$ sowie die dem Moment entsprechende Fläche in der komplexen Raumzeigerebene, wenn die Strangachse "U" in der reellen Achse liegt. (\addpoints{2} + \addpoints{1} für die Skizze)
		\item Berechnen Sie für den Zeitpunkt $\tau_0$ (statorfeste Rotorverkettungsfluß $\underline{\Psi}_{M,(\tau_0)} = 1\cdot e^{\jmath 40^\circ}$) im \textbf{Sinusbetrieb} jeweils den optimalen normierten Stromraumzeiger im statorfesten Koordinatensystem, um \underline{halbes motorisches und gernatorisches} Drehmoment zu erzeugen. Geben Sie für beide Fälle die nichtbezogenen Ströme $I_1$,$I_2$,$I_3$ an. (\addpoints{2})
		\item Berechnen Sie für den Statorkurzschluss $\underline{u}_s =0$ allgemein den Verlauf des stationären Kurzschlussmoments $m(\omega_m)$ in Abhängigkeit der Drehzahl $\omega_m$, des Statorwiderstandes $r_s$ und der Statorinduktivität $l_s$. Berechnen Sie daraus die Statorinduktivität $l_s$, wenn das Maximum des Kurzschlussmoments bei einer Drehzahl $\omega_m = 0,15$ liegt und der Statorwiderstand $r_s = 0,045$ beträgt. Wie groß ist das max. Kurzschlussmoment $m(\omega_m = 0,15)$? Skizzieren Sie den Verlauf $m(\omega_m)$ im Bereich $\omega_m =-1$...$+1$. (\addpoints{3})
	\end{compactenum}
\end{question}
\begin{solution}
	
\end{solution}