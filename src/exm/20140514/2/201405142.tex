\begin{question}[topic=gsm,name={14.5.2014},type=exam,tags={20140514}]
Eine permanentmagneterregte Gleichstrommaschine hat folgende Daten.\\
\begin{tabular}{L{2cm}l}
$I_{A,N}$ \dotfill &$10~A$\\
$U_{A,N}$ \dotfill & $48~V$ \\
$n_0$ \dotfill & $4000~\frac{U}{min}$
\end{tabular}\\
\begin{enumerate}
\item Wie groß ist die Spannungskonstante $k_1 \cdot \phi$ und das Nennmoment $M_N$ der Gleichstrommaschine? (\addpoints{2})
\item Berechnen Sie den Ankerwiderstand $R_A$ der Gleichstrommaschine, wenn durch eine Messung bei $U_A = 12~V$ unter Belastung mit Nennstrom $I_{A,N}$ eine Drehzahl $n= 750~ U/min$ ermittelt wurde. Berechnen Sie weiters die Nenndrehzahl $n_N$ und den Wirkungsgrad der Gleichstrommaschine im Nennpunkt ($M_N$, $n_N$). (\addpoints{2})
\item Die Gleichstrommaschine (Ankerspannung $U_A$) wird mittels eines idealen Tiefsetzstellers von einer Batterie $U_B = 48~V$ versorgt. Wie groß ist das Tastverhältnis $\delta = U_A / U_B$ zu wählen, damit bei halber Leerlaufdrehzahl eine mechanische Leistung von $P_{mech} = 200~W$ abgegeben wird? (\addpoints{2})
\item Aufgrund einer Erwärmung der Gleichstrommaschine um $50~ ^\circ C$ kommt es zur Veränderung des Erregerflusses sowie des Ankerwiderstandes ausgehend von den angegebenen Nenndaten. Die Temperaturabhängigkeit des Ankerwiderstands wird mit $R_T = R_0 \cdot ( 1 + \alpha \cdot(T - T_0))$ und einem Temperaturkoeffizient $\alpha = 0,00393~ K^{-1}$ berücksichtigt. Im Datenblatt des eingebauten Permanentmagneten wird der Temperaturkoeffizient der Remanenzflussdichte $B_r$ mit $-0,2\%/K$ angegeben welche gleichermaßen für die Beschreibung der Temperaturabhängigkeit des Erregerflusses $\phi(T)$ verwendet werden soll. Berechnen Sie für die erhöhte Temperatur die Spannungskonstante, den Ankerwiderstand, sowie die Nenndrehzahl und Nennleistung bei unverändertem Nennstrom und einer Ankerspannung $U_A = 48~V$. (\addpoints{2})
\end{enumerate}
\end{question}
\begin{solution}
\begin{enumerate}
\item Im Leerlauf ist der Ankerstrom Null und mittels Glg.(\ref{glg:Ankerspannungsgleichung}) lässt sich die Spannungskonstante errechnen.
\begin{align}
U_{A,N} &= \frac{k_1 \cdot \phi}{2 \pi} \cdot \frac{4000}{60} \cdot 2 \pi\\
k_1 \cdot \phi &= 0,72~Vs\\
k^{'} \cdot \phi &= \frac{k_1 \cdot \phi}{2 \pi} = 0,144~Vs\\
M_N &= \frac{k_1 \cdot \phi_N}{2 \pi} \cdot I_A = 1,145~Nm
\end{align}
\item Die Glg.(\ref{glg:Ankerspannungsgleichung}) wird auf $R_A$ umgeformt. Für die Berechnung der Nenndrehzahl wird in Glg.(\ref{glg:Ankerspannungsgleichung}) die Nennwerte eingesetzt und auf $\Omega_N$ umgeformt. Um auf $n_N$ zu kommen wird noch mit $\frac{60}{2 \pi}$ mulitpliziert. Der Wirkungsgrad errechnet sich nach Glg.(\ref{glg:Wirkungsgrad}).
\begin{align}
U_{A,N} &= R_A I_A + k^{'}\cdot \phi \cdot \frac{n_0 2 \pi}{60}\\
R_A &= \frac{U_{A,N} -k^{'}\cdot \phi \cdot \frac{n_0 2 \pi}{60}}{I_A} = 300~m\Omega\\
\Omega_N &= \frac{U_{A,N} - R_A I_A}{k^{'}\cdot \phi}\\
n_N &= \frac{U_{A,N} - R_A I_A}{k^{'}\cdot \phi} \cdot \frac{60}{2 \pi} = 3750 ~U /min\\
\eta_N &= \frac{\Omega_N M_N}{U_N I_N} = 0,938
\end{align}
\item In Glg.(\ref{glg:mechanischleistung}) wird Glg.(\ref{glg:Ankerspannungsgleichung}) auf $I_A$ umgeformt und eingesetzt. Anschließend wird auf $\delta$ umgeformt.
\begin{align}
P_{mech} &=M_m \cdot \Omega_m = k^{'} \phi \cdot I_A \cdot \Omega_m\\
&= k^{'} \phi \cdot \frac{U_B \delta - k^{'} \phi \Omega}{R_A} \cdot \Omega\\
\delta &= \frac{\frac{P_{mech}}{\Omega} \cdot R_A + (k^{'} \phi)^2 \Omega}{k^{'} \phi U_B}=0,552
\end{align}
\item Da die Spannungskonstante wird über den Dauermagneten beschrieben.
\begin{align}
k_1 \cdot \phi_{N,50^\circ C}&= k_1 \phi_N \cdot (1 - 0,002 \cdot 50)= 0,648\\
R_{A,50^\circ C} &= 0,3 \cdot (1+ 0,00393 \cdot 50) = 359~m\Omega\\
n_{N,50^\circ C} &= \frac{U_A -R_{A,50^\circ C} \cdot I_A}{k_1 \cdot \phi_{N,50^\circ C}}\cdot \frac{60}{2 \pi} = 4112 ~U/min
\end{align}
\end{enumerate}
\end{solution}