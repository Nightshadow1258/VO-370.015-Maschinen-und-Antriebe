\begin{question}[topic=psm,type=exam,name={9.5.2011},tags={20110509}]
	Eine kurzgeschlossene permanentmagneterregte Synchronmaschine wird schlagartig an eine sich mit konstanter Drehzahl drehende Welle gekuppelt $n_{m,t=0}=0$ , $n_{w,t=0} = \omega_w$. Die Maschine war vor dem Kuppelvorgang stromlos.
	\begin{enumerate}
		\item Geben Sie die Gleichungen für die Raumzeiger der Statorspannung und des Statorverkettungsflußes an. (\addpoints{2})
		\item Um den Verlauf des Statorstromzeigers Berechnen zu können, geben Sie zuerst die Übertragungsfunktion des Statorstromraumzeigers an (Statorfest). \textit{Hinweis:} Als anregende Spannung wirkt die Induzierte Spannung, wobei Sie den Lagewinkel $e^{\jmath \gamma}$ durch $e^{\jmath \gamma} \cdot e^{\jmath \omega t}$ ausdrücken können. (\addpoints{4})
		\item Berechnen Sie den zeitlichen Verlauf des Statorstromraumzeigers im statorfesten Koordinatensystem. (\addpoints{2})
		\item Skizzieren Sie den Verlauf des Statorstromraumzeigers (statorfestes Koordinatensystem) in der Stromraumzeigerebene für eine elektrische Umdrehung, einmal unmittelbar nach dem Kurzschluss und einmal für den eingeschwungenen Zustand. (\addpoints{2})
	\end{enumerate}
\end{question}
\begin{solution}
	\textbf{Hinweis:} Diese Prüfung ist noch aus dem Masterstudium und ist nicht repräsentativ für den Prüfungsstoff im Bachelorstudium.
\end{solution}