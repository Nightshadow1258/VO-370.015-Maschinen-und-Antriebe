\begin{question}[topic=psm,type=exam,tags={20131203}]
Eine dreisträngige symetrisch aufgebaute permanentmagneterregte Synchronmaschine läuft mit eingeprägter positiver Drehzahl von $10\%$ der Bezugsdrehzahl. Zum Zeitpunkt $\tau_0$ ist der normierte statorfeste Rotorverkettungsfluss $\Psi_M = 1 \cdot e^{\jmath \cdot 50 ^\circ}$.
\begin{enumerate}
\item Berechnen Sie für einen BLDC-Betrieb jenen günstigen normierten Stromraumzeiger zum Zeitpunkt $\tau_0$, welcher das halbe generatorische Bezugsmoment bei positiver Drehrichtung ergibt. Geben Sie ebenfalls die bezogenen Ströme in den Motorzuleitungen an. (\addpoints{2})
\item Berechnen Sie für einen Sinus-Betrieb jenen normierten Stromraumzeiger, welcher das gleiche Drehmoment unter optimaler Drehmomentausnutzung bei positiver Drehrichtung ergibt. Wie groß ist das Verhältnis des Strombetrags zum BLDC Betrieb für diesen Zeitpunkt $\tau_0$? (\addpoints{2})
\item Skizzieren Sie $\underline{\Psi}_M$,$\underline{i}_s$ und die dem Drehmoment entsprechende Fläche in der komplexen Raumzeigerebene (Strangachse "`U"' liegt in der reellen Achse) für beide obigen Betriebspunkte. (\addpoints{2})
\item Berechnen Sie für den stationären Sinus-Betrieb den bezogenen Statorspannungsraumzeiger im rotorfesten Koordinatensystem zum Zeitpunkt $\tau_0$ mit den Maschinendaten $r_s = 0,05$ und $l_s = 0,25$. (\addpoints{1})
\item Ausgehend vom Sinusbetrieb wird zum Zeitpunkt $\tau_0$ bei eingeprägter Drehzahl vom Spannungszwischenkreisumrichter ein Klemmenkurzschluss angelegt. Berechnen Sie für den Zeitpunkt $\tau_0$ die bezogene Stromänderung $\partial \underline{i}_s/\partial \tau$ sowie die Momentenänderung $\partial m/\partial \tau$. (\addpoints{2})
\item Berechnen Sie den stationären Stromraumzeiger $ \tau \rightarrow \infty$ im rotorfesten Koordinatensystem für die kurzgeschlossene Maschine bei konstanter Drehzahl, sowie das stationäre Drehmoment. (\addpoints{3})
\end{enumerate}
\end{question}
\begin{solution}
\begin{enumerate}
\item Das halbe generatorische Bezugsmoment bei positiver Drehzahl bedeutet $m_m = -0,5$. Im BLDC Betrieb ist der n\"achste Stromzeiger gem Abb.(\ref{fig:bldc}) zu $-40^\circ$ bei $-30^\circ$ der Fall A. \"Uber die Dreiecksbeziehungen werden $\underline{i}_{sdq}$ und $\underline{i}_{sd}$ aus $\underline{i}_{sq}$ berechnet. Anschließend wird $\underline{i}_{sdq}$ auf $\underline{i}_{s}$ umgeformt um die Str\"ome in den Motorzuleitungen zu berechnen.
\begin{align}
m_m &= -0,5 = \underline{i}_{sq} \cdot |\underline{\Psi_m}|\\
-0,5 &= \underline{i}_{sq} \cdot 1\\
\underline{i}_{sq} &= -0,5\\
\arg(\underline{i}_{sdq}) &= \arg(\underline{i}_{s}) -\arg(\underline{\Psi}_{M})=-80^\circ\\
\underline{i}_{sdq} &= \frac{\underline{i}_{sq}}{\sin(\arg(\underline{i}_{sdq}))}= 0,507\\
\underline{i}_{sd} &= \underline{i}_{sdq} \cdot \cos(\arg(\underline{i}_{sdq})) = 0,088\\
\underline{i}_{s} &= |\underline{i}_{sdq}| \cdot e^{\jmath (\arg(\underline{i}_{sdq}) + \arg(\underline{\Psi}_{M}))}= 0,507 \cdot e^{\jmath ( -80 + (50))}
\end{align}
In die Glg.(\ref{glg:strang1}),(\ref{glg:strang2}) und (\ref{glg:strang3}) wird der Statorstrom $\underline{i}_s$ eingesetzt.
\begin{align}
i_1 & = \Re \{ \underline{i}_s \cdot e^{\jmath \cdot 0 ^\circ} \} = 0,44\\
i_2 & = \Re \{ \underline{i}_s \cdot e^{-\jmath \cdot 120 ^\circ} \} = -0,44 \\
i_3 & = \Re \{ \underline{i}_s \cdot e^{\jmath \cdot 120 ^\circ} \}=  0
\end{align}
Wie in Tab.(\ref{tab:bldc}) ersichtlich muss f\"ur den Fall A $i_1= -i_2$ sein und $i_3= 0$ gelten.
\item Da wir uns im Sinus-Betrieb befinden ist der Stromraumzeiger $\underline{i}_{sq}$, welche das optimale Bezugsdrehmoment liefert, gleich dem Stromraumzeiger $\underline{i}_{sdq}$. Der Winkel liegt somit exakt bei $\arg(\underline{i}_{sdq})=-90^\circ$ zu $\underline{\Psi}_M$.
\begin{align}
m_m &= -0,5 = \underline{i}_{sq} \cdot |\underline{\Psi_m}|\\
-0,5 &= \underline{i}_{sq} \cdot 1\\
\underline{i}_{sq} &= -0,5 = \underline{i}_{sdq} \\
\underline{i}_{sd} &= 0\\
\underline{i}_{s} &= |\underline{i}_{sdq}| \cdot e^{\jmath (\arg(\underline{i}_{sdq}) + \arg(\underline{\Psi}_{M}))}= 0,5 \cdot e^{\jmath ( -90 + 50)}
\end{align}
\item TikZ Grafik
\item Der Spannungraumzeiger im rotorfesten Koordinatensystem errechnet sich nach Glg.(\ref{glg:Statorspannungraumzeiger rotor}), wobei die partiellen Terme wegfallen, weil wir uns im station\"aren Fall befinden. F\"ur $\underline{i}_{sdq}= -0,5 \jmath$ einsetzen.
\begin{align}
\underline{u}_{sdq}(\tau) &= \underline{i}_{sdq} \cdot r_s + l_s \cdot \frac{\partial \underline{i}_{sdq}}{\partial \tau} + \frac{\partial |\underline{\Psi}_M|}{\partial \tau} + \jmath \omega_K \cdot l_s \cdot \underline{i}_{sdq} + \jmath \omega_K \cdot |\underline{\Psi}_M|\\
\underline{u}_{sdq}(\tau) &= \underline{i}_{sdq} \cdot r_s + \jmath \omega_K \cdot l_s \cdot \underline{i}_{sdq} + \jmath \omega_K \cdot |\underline{\Psi}_M|\\
&= -0,5 \jmath \cdot 0,05 + \jmath 0,1 \cdot 0,25 \cdot -0,5 \jmath+\jmath 0,1 \cdot 1\\
&=0,076 \cdot e^{\jmath 99,46^\circ}
\end{align}
\item Der Strom wird durch die Spule gehalten, deshalb ist $\partial \frac{\underline{i}_s}{\partial \tau} = \frac{m}{\partial \tau} = 0$
\item Die Herleitung zu Glg.(\ref{glg:stromzeiger rotorfest ks}) durchf\"uhren und dann in Glg.(\ref{glg:synmoment}) einsetzen.
\begin{equation}
i_{sq} \cdot | \underline{\Psi}_M| = -\frac{\omega |\underline{\Psi}_M|^2 r_s}{r_s^2 + (\omega l_s)^2}
\end{equation}
\end{enumerate}
\end{solution}