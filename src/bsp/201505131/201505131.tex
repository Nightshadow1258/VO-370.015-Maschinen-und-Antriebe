\begin{question}[topic=psm,type=exam,tags={20150513}]
Eine dreisträngige symetrische aufgebaute permanentmagneterregte Synchronmaschine in Sternschaltung ohne Mittelpunktleiter ($I_N = 10~A$, $U_N= 230~V$) läuft mit eingeprägter positiver Drehzahl von \textbf{30\% der Bezugsdrehzahl}. Zum Zeitpunkt $\tau_0$ ist der normierte statorfeste Rotorverkettungsfluss $\underline{\Psi}_M = 1 \cdot e^{\jmath \cdot 80 ^\circ}$.
\begin{enumerate}
\item Berechnen Sie für einen BLDC-Betrieb jenen günstigen normierten Stromraumzeiger, welche das \textbf{halbe generatorische} Bezugsmoment bei positiver Drehzahl ergibt. Geben Sie ebenfalls die bezogenen und nicht bezogenen Ströme in den Motorzuleitungen an. (\addpoints{2})
\item Berechnen Sie für einen Sinus-Betrieb jenen normierten Stromraumzeiger, welcher das gleiche \textbf{(halbe generatorische)} Drehmoment unter optimaler Drehmomentausnutzung bei positiver Drehrichtung ergibt. Wie groß ist das Verhältnis der Strombetrags zum BLDC-Betrieb für diesen Zeitpunkt? (\addpoints{2})\label{2015051312}
\item Skizzieren Sie jeweils die Raumzeiger $\underline{\Psi}_M$,$\underline{i}_s$ und die dem Moment entsprechende Fläche in der komplexen Raumzeigerebene (Strangachse ``U'' liegt in der reellen Achse) für beide obrigen Betriebpunkte. (\addpoints{1} +\addpoints{1} Skizze)
\item Berechnen Sie für den stationären Sinus-Betrieb (siehe Punkt \ref{2015051312}.) den bezogenen Statorspannungsraumzeiger im rotorfesten und statorfesten Koordinatensystem zum Zeitpunkt $\tau_0$ mit den Maschinendaten $r_s = 0,08$ und $l_s = 0,3$. (\addpoints{2})
\item Berechnen Sie für die Maschinendaten $r_s=0,08$ und $l_s = 0,3$ den Verlauf des stationären Kurzschlussmoments $m(\omega_m)$ in Abhängigkeit der Drehzahl $\omega_m$ für einen kurzgeschlossenen Stator und skizzieren Sie den Verlauf. (\addpoints{2})
\end{enumerate}
\end{question}
\begin{solution}
\begin{enumerate}
\item Das \textbf{halbe} generatorische Bezugsmoment bei positiver Drehzahl bedeutet $m_m = -0,5$. Im BLDC Betrieb ist der n\"achste Stromzeiger gem Abb.(\ref{fig:bldc}) zu $80^\circ$ bei $-30^\circ$ der Fall A. \"Uber die Dreiecksbeziehungen werden $\underline{i}_{sdq}$ und $\underline{i}_{sd}$ aus $\underline{i}_{sq}$ berechnet. Anschließend wird $\underline{i}_{sdq}$ auf $\underline{i}_{s}$ umgeformt um die Str\"ome in den Motorzuleitungen zu berechnen.
\begin{align}
m_m &= -0,5 = \underline{i}_{sq} \cdot |\underline{\Psi_m}|\\
-0,5 &= \underline{i}_{sq} \cdot 1\\
\underline{i}_{sq} &= -0,5\\
\arg(\underline{i}_{sdq}) &= \arg(\underline{i}_{s}) -\arg(\underline{\Psi}_{M})=-110^\circ\\
\underline{i}_{sdq} &= \frac{\underline{i}_{sq}}{\sin(\arg(\underline{i}_{sdq}))}= 0,532\\
\underline{i}_{sd} &= \underline{i}_{sdq} \cdot \cos(\arg(\underline{i}_{sdq})) = -0,181\\
\underline{i}_{s} &= |\underline{i}_{sdq}| \cdot e^{\jmath (\arg(\underline{i}_{sdq}) + \arg(\underline{\Psi}_{M}))}= 0,532 \cdot e^{\jmath ( -110 + (80))}
\end{align}
In die Glg.(\ref{glg:strang1}),(\ref{glg:strang2}) und (\ref{glg:strang3}) wird der Statorstrom $\underline{i}_s$ eingesetzt.
\begin{align}
i_1 & = \Re \{ \underline{i}_s \cdot e^{\jmath \cdot 0 ^\circ} \} = 0,46\\
i_2 & = \Re \{ \underline{i}_s \cdot e^{-\jmath \cdot 120 ^\circ} \} = -0,46 \\
i_3 & = \Re \{ \underline{i}_s \cdot e^{\jmath \cdot 120 ^\circ} \}=  0
\end{align}
Wie in Tab.(\ref{tab:bldc}) ersichtlich muss f\"ur den Fall A $i_1= -i_2$ sein und $i_3= 0$ gelten.
Um die nicht bezogenen Ströme zu erhalten werden die bezogenen Ströme mit dem Bezugswert $I_N \cdot \sqrt{2}$ multipliziert. (Effektivwert auf Spitzenwert umrechnen)
\begin{align}
I_1 & = i_1 \cdot I_N \cdot \sqrt{2} = 0,46\cdot 10 A \cdot \sqrt{2} =6,517~A \\
I_2 & = i_2 \cdot I_N \cdot \sqrt{2} = -0,46 \cdot 10 A \cdot \sqrt{2} =-6,517~A \\
I_3 & = i_3 \cdot I_N \cdot \sqrt{2} =0 \cdot 10 A \cdot \sqrt{2} =0~A
\end{align}
\item Da wir uns im generatorischen linksdrehenden Sinus-Betrieb befinden ist der Stromraumzeiger $\underline{i}_{sq}$, welche das optimale \textbf{halbe} Bezugsdrehmoment liefert, gleich dem Stromraumzeiger $\underline{i}_{sdq}$. Der Winkel liegt somit exakt bei $\arg(\underline{i}_{sdq})=-90^\circ$ zu $\underline{\Psi}_M$.
\begin{align}
m_m &= -0,5 = \underline{i}_{sq} \cdot |\underline{\Psi_m}|\\
-0,5 &= \underline{i}_{sq} \cdot 1\\
\underline{i}_{sq} &= -0,5 = \underline{i}_{sdq} \\
\underline{i}_{sd} &= 0\\
\underline{i}_{s} &= |\underline{i}_{sdq}| \cdot e^{\jmath (\arg(\underline{i}_{sdq}) + \arg(\underline{\Psi}_{M}))}= -0,5 \cdot e^{\jmath (-90 + 80)}
\end{align}
In die Glg.(\ref{glg:strang1}),(\ref{glg:strang2}) und (\ref{glg:strang3}) wird der Statorstrom $\underline{i}_s$ eingesetzt.
\begin{align}
i_1 & = \Re \{ \underline{i}_s \cdot e^{\jmath \cdot 0 ^\circ} \} = 0,492\\
i_2 & = \Re \{ \underline{i}_s \cdot e^{-\jmath \cdot 120 ^\circ} \} = -0,321 \\
i_3 & = \Re \{ \underline{i}_s \cdot e^{\jmath \cdot 120 ^\circ} \}=  -0,171
\end{align}
Um die nicht bezogenen Ströme zu erhalten werden die bezogenen Ströme mit dem Bezugswert $I_N \cdot \sqrt{2}$ multipliziert. (Effektivwert auf Spitzenwert umrechnen)
\begin{align}
I_1 & = i_1 \cdot I_N \cdot \sqrt{2}  =6,958~A \\
I_2 & = i_2 \cdot I_N \cdot \sqrt{2} =-4,545~A \\
I_3 & = i_3 \cdot I_N \cdot \sqrt{2} =-2,418~A
\end{align}
Das Verh\"altnis der Strombelags ist der Betrag $I_n/ I_{ZK}$, wobei $I_{ZK}$ gleich dem Strom im BLDC ist.
\begin{align}
\alpha_1 = \frac{|I_1|}{|I_{ZK}|}= 1,068\\
\alpha_2 = \frac{|I_2|}{|I_{ZK}|}= 0,697\\
\alpha_3 = \frac{|I_3|}{|I_{ZK}|}= 0,371
\end{align}
\item TikZ Grafik hier.
\item Der Spannungraumzeiger im rotorfesten Koordinatensystem errechnet sich nach Glg.(\ref{glg:Statorspannungraumzeiger rotor}), wobei die partiellen Terme wegfallen, weil wir uns im station\"aren Fall befinden. F\"ur $\underline{i}_{sdq}= 0,5 e^{\jmath -90^\circ}$ einsetzen.
\begin{align}
\underline{u}_{sdq}(\tau) &= \underline{i}_{sdq} \cdot r_s + l_s \cdot \frac{\partial \underline{i}_{sdq}}{\partial \tau} + \frac{\partial |\underline{\Psi}_M|}{\partial \tau} + \jmath \omega_K \cdot l_s \cdot \underline{i}_{sdq} + \jmath \omega_K \cdot |\underline{\Psi}_M|\\
\underline{u}_{sdq}(\tau) &= \underline{i}_{sdq} \cdot r_s + \jmath \omega_K \cdot l_s \cdot \underline{i}_{sdq} + \jmath \omega_K \cdot |\underline{\Psi}_M|\\
&= 0,5 e^{\jmath -90^\circ} \cdot 0,08 + \jmath 0,3 \cdot 0,3 \cdot 0,5 e^{-\jmath 90^\circ}+\jmath 0,3 \cdot 1\\
&=0,264 \cdot e^{\jmath 80,18^\circ}
\end{align}
Der Spannungraumzeiger im statorfesten Koordinatensystem errechnet sich nach Glg.(\ref{glg:spannungraumzeiger rotor}). F\"ur $\underline{i}_{s}= 0,5 e^{\jmath -10^\circ+\omega \tau}$ einsetzen. Da $\tau_0 = 0$ ist, f\"allt der zweite teil weg.
\begin{align}
\underline{u}_s(\tau) &= \underline{i}_s \cdot r_s + l_s \cdot \frac{\partial \underline{i}_s}{\partial \tau} + \jmath\omega \cdot|\underline{\Psi}_M | \cdot e^{\jmath \cdot \gamma + \jmath \omega \tau}\\
\underline{u}_s(\tau) &= \underline{i}_s \cdot r_s + \jmath\omega \cdot|\underline{\Psi}_M | \cdot e^{\jmath \cdot \gamma}=0,26\cdot e^{\jmath 169,6^\circ}
\end{align}
\item In Glg.(\ref{glg:synmoment}) wird der Imagin\"arteil von Glg.(\ref{glg:stromzeiger rotorfest ks}) eingesetzt.
\begin{align}
m_m(\omega_m) &= \frac{\omega |\underline{\Psi}_M|^2 r_s}{r_s^2 + (\omega l_s)^2}=\frac{0,888 \cdot \omega}{\omega^2 + 0,0711}
\end{align}
TikZ Grafik hier.
\end{enumerate}
\end{solution}