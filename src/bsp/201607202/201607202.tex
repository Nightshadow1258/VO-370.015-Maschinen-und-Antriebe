\begin{question}[topic=gsm,type=exam,tags={20160720}]
Eine kompensierte \textbf{Reihenschluss-Gleichstrommaschine} hat folgende Daten. Die Maschine ist im Nennpunkt nicht ges\"attigt ($\Phi \sim I_A$). Die Maschine hat nur ohmsche Verluste, Reibungsverluste und Eisenverluste sind vernachlässigbar.
\begin{tabular}{L{2cm}l}
$I_{A,N}$ \dotfill &$225~A$\\
$P_{N,mech}$ \dotfill & $81~kW$ \\
$n_N$ \dotfill & $2000~\frac{U}{min}$\\
$\eta_N$ \dotfill & $90\%$
\end{tabular}
\begin{enumerate}
\item Skizzieren Sie die Schaltung der Reihenschlussmaschine am Gleichspannungsnetz inkl. aller Widerst\"ande und Induktivit\"aten der Maschine. (\addpoints{1})
\item Wie groß ist das Nennmoment $M_N$ und die Spannungskonstante $k_1 \cdot \phi_N$ im Nennpunkt im motorischen Betrieb? Wie groß ist die Leerlaufdrehzahl $n_0$? (\addpoints{2})
\item Wie groß ist die Nennspannung $U_N$ und wie groß ist der Ankerwiderstand $R_A$ und der Erregerwiderstand $R_E$, wenn sich der Erregerwiderstand $R_E$ zum Ankerwiderstand $R_E:R_A=2:5$ verh\"alt? (\addpoints{2})
\item Skizzieren Sie ma\ss st\"ablich die Drehzahl-Drehmoment Kennlinie (M/n) bei Nennspannung im Bereich ca. $0,2 \cdot M_N$ bis $1,5 \cdot M_N$. (\addpoints{2})
\item Berchnen Sie den ben\"otigten Vorwiderstand $R_V$ wenn die Maschine bei Nennspannung $U_N$ mit $M=1,5 \cdot M_N$ aus dem Stillstand angefahren werden soll. (\addpoints{1})
\item Die Gleichstrommaschine wird von einem Stromrichter mit konstantem, halben Nennstrom bei $n = 2000 ~U/min$ als Motor betrieben. Berechnen und Skizzieren Sie den Drehzahlverlauf $n(t)$, wenn die Last schlagartig abgekuppelt wird. Der Stromrichter liefert dabei weiterhin den konstanten Strom und schaltet die Gleichstrommaschine erst bei Erreichen einer Spannung von $U= 460~V$ ab. Wie lange dauert es bis zum Abschalten und welche Enddrehzahl wird erreicht? Das Tr\"agheitsmoment der Gleichstrommaschine ist $\Theta_{GM} = 12 ~kg~m^2$ und das Reibungsmoment betr\"agt konstant $1\%$ des Nennmoments. (\addpoints{2})
\end{enumerate}
\end{question}
\begin{solution}
\begin{enumerate}
\item TikZ Grafik hier.
\item Die Spannungskonstante errechnet sich durch umformen der Glg.(\ref{glg:mechanischleistung}). Durch umformen der Glg.(\ref{glg:Ankerspannungsgleichung}) auf $\Omega$ und anschließendendem mulitplizieren mir $\frac{60}{2 \pi}$ erhalten wir die Leerlaufdrehzahl. Die Ankerspannung ist dabei $\frac{P_{mech}}{0,9 I_{A,N}}$
\begin{align}
P_{mech} &=M_m \cdot \Omega_m = k^{'} \phi \cdot I_A \cdot \Omega_m\\
k^{'} \phi &= \frac{P_{mech}}{I_A \cdot \Omega_m} =1,718~Vs\\
M_N &= k^{'} \phi \cdot I_A = 386,75~Nm\\
U_{A,N} &= \frac{P_{mech}}{0,9 I_{A,N}} =400~V\\
n_0 &= \frac{U_{A,N}}{k^{'} \phi} \cdot \frac{60}{2\pi} =2222~U/min
\end{align}
\item Ankerspannung ist im vorherigen Unterpunkt schon berechnet worden. Die Widerst\"ande errechnen sich aus Glg.(\ref{glg:Ankerspannungsgleichung}) durch umformen und einsetzen der Widerstandsbeziehung.
\begin{align}
U_{A,N} &= (R_E + R_A) \cdot I_A +  k^{'} \phi \cdot \Omega_N\\
U_{A,N} &= \left (\frac{2}{5} + 1 \right ) \cdot R_A \cdot I_A +  k^{'} \phi \cdot \Omega_N\\
R_A &= \frac{U_{A,N}-k^{'} \phi \cdot \Omega_N}{I_A \cdot \left (\frac{2}{5} + 1 \right )}=126,98~m\Omega\\
R_E &= \frac{2}{5} \cdot R_A= 50,79~m\Omega
\end{align}
\item Das Ankermoment (\ref{glg:Ankermoment}) wird auf $I_A$ umgeformt und in (\ref{glg:Ankerspannungsgleichung}) eingesetzt und auf $\Omega$ umgeformt und Anschließend mit $\frac{60}{2 \pi}$ mulitpliziert um auf $n$ zu kommen.
\begin{equation}
n(M_i) = \frac{U_A - (R_A+R_E) \frac{ M_i}{k^{'} \Phi}}{k^{'}\Phi} \cdot \frac{60}{2 \pi} =2222-0,575 \cdot M_i
\end{equation}
TikZ Grafik.\\
\item Keine Ahnung.
\item Keine Ahnung.
\end{enumerate}
\end{solution}